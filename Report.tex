\documentclass{report} % Add the document class
\usepackage{graphicx} % For including graphics like logos
\usepackage{lipsum}   % For dummy text
\usepackage{setspace} % For line spacing
\usepackage{fancyhdr} % For custom headers and footers
\usepackage{geometry} % For page margins
\usepackage{amsmath} % For mathematical equations
\usepackage{enumitem} % For customized lists
\usepackage{amsfonts} % For the \forall symbol
\usepackage{multicol} % For multiple columns if needed
\usepackage{amsmath} % For additional math formatting
\usepackage{enumitem} % For customizing lists
\usepackage{acronym} % For defining acronyms
\usepackage{hyperref} % For hyperlinks for table of contents
\usepackage{tocbibind} % For adding list of figures, tables to table of contents
\geometry{top=1in, bottom=1in, left=1in, right=1in} % Set page margins

% Configure the hyperref package to remove red boxes and customize link colors
\hypersetup{
    colorlinks=true,      % Set to true to enable colored links
    linkcolor=black,       % Color for internal links (sections, pages, etc.)
    citecolor=black,       % Color for citation links
    filecolor=black,       % Color for file links
    urlcolor=black         % Color for URL links
}

\begin{document}

% Set up the header and footer using fancyhdr
% \pagestyle{fancy}
% \fancyhf{} % Clear all header and footer fields

% % Define the header
% \fancyhead[L]{
%     \small
%     Technical University of Applied Sciences Würzburg-Schweinfurt (THWS)\\
%     Faculty of Computer Science and Business Information Systems
% }

% % Adjust the header position
% \renewcommand{\headrulewidth}{0pt} % Remove the header rule line


% Title Page
\begin{titlepage}
    \centering
    \vspace*{1cm}
    
    \Large \textbf{Technical University of Applied Sciences Würzburg-Schweinfurt (THWS)}\\
    \vspace{0.5cm}
    \Large Faculty of Computer Science and Business Information Systems\\
    \vspace{1cm}
    
    \huge \textbf{Master Thesis}\\
    \vspace{1.5cm}
    
    \Huge \textbf{Electric Motor Modelling via Graph Neural Networks}\\
    \vspace{2cm}
    
    \large \textbf{Submitted to the Technical University of Applied Sciences Würzburg-Schweinfurt in the Faculty of Computer Science and Business Information Systems to
    complete a course of studies in Master of Artificial Intelligence}
    
    \vspace{1cm}
    
    \huge Lilly Abraham\\
    \huge K64889\\
    \vspace{1cm}
    \large To be Submitted on: 11.12.2024\\ % replace with Submitted on
    
    \vfill
    
    \large
    Initial examiner: Prof. Dr. Magda Gregorova\\
    Secondary examiner: Prof. Gracia Herranz Mercedes\\

\end{titlepage}

\newpage % Start a new page


% Including an image on this page
\begin{figure}[h]
    \includegraphics[width=0.8\textwidth]{./ReportImages/qrcode.png} % Adjust path and filename
    \label{fig:your-image}
\end{figure}

\newpage % Start a new page

\chapter*{Abstract}
\addcontentsline{toc}{chapter}{Abstract}

The thesis explores an approach to predict KPIs of topology invariant IPSM Electric Motors by transforming its geometric, physical and simulation parameters into a graph representation. \\
The KPIs to be predicted are plots on Efficiency grid(3D) and Torque curve(2D).\\
We aim to first parameterize the EM design such that it is feasible to convert into a graph representation. \\
Next, we would create a Graph with relevant attributes and design a GNN with the graph as input and the plots in the format of vectors as target values.\\
Additionally we may also need to customize the loss function in a way that would smoothen out the plot curves of the prediction values.\\
Then, we would evaluate the predictions with the test target values by experimenting with various hyperparameter tuning settings and as a baseline with an MLP model of the parameters in tabular form.\\
Finally we will enable the KPI's plot visualisation in a manner presentable to the client Valeo(Automaker Company).\\
Not necessary remove i suppose....
The aim of the Master Thesis is to train a neural network to learn the parameters of Electric Motors and thus be able to predict its KPIs.
The KPIs are 2D and 3D plots on Torque(Mgrenz) curve(Mgrenz) and Efficiency grid(ETA). Other KPIs can be calculated from these two KPIs.
For instance the Vibration Costs are inversely proportional to the Efficieny values predicted. 


\newpage 

\newpage 

\chapter*{Acknowledgement}
\addcontentsline{toc}{chapter}{Acknowledgement}
I would like to thank my supervisor Prof. Dr. Magda Gregorova for her guidance and support throughout the course of this thesis and Valeo for providing the data.
Special thanks to Mr Daniel and Leo for sharing valuable insights of the data from an electromechanincal standpoint.

\newpage

\newpage

\begin{spacing}{1.2}
    \tableofcontents
\end{spacing}

\newpage

\newpage

\chapter*{Abbreviations}
\addcontentsline{toc}{chapter}{Abbreviations}
\begin{acronym}[TDMA]
  
    \acro{GNN :}{Graph Neural Network}
    \acro{MLP :}{Multi Linear Perceptron}
    \acro{GNN :}{Graph Neural Network}
    \acro{KPI :}{Key Performance Indicator}
    \acro{EM :}{Electric Motor}
    \acro{FEM :}{Finite Element Method}
    \acro{CNN :}{Convolution Neural Network}
    \acro{2D :}{2 Dimension}
    \acro{3D :}{3 Dimension}

\end{acronym}

\newpage

\newpage

\chapter*{Introduction} 
\addcontentsline{toc}{chapter}{Introduction}
In the design of electric motors, vast amounts of data are generated to determine which design of an EM fits best to KPIs. \\
KPIs of an Electric Motor are essential to judge the performance of the motor before it is manufactured. \\
Traditionally these KPIs are inferred from a description of an EM design via a FEM approximating the solutions of the Maxwell’s equations. This process, though well established in the EM design, is very time consuming and does not allow for high-throughput engine design optimization. \\
The actual engine data of Valeo is used here as the dataset comprising of multiple variant designs of the Double-V topology.\\
The 3 motor topologies manufactured by Valeo are as below:


\begin{figure}[h]
    \centering
    \begin{minipage}[b]{0.3\textwidth}
        \includegraphics[width=\textwidth]{./ReportImages/1V_Magnet.png}
        \caption{V1 Magnet \\ (Source:Valeo)}
        \label{fig:V1 Magnet}
    \end{minipage}
    \hfill
    \begin{minipage}[b]{0.3\textwidth}
        \includegraphics[width=\textwidth]{./ReportImages/2V_Magnet.png}
        \caption{V2 Magnet\\ (Source:Valeo)}
        \label{fig:V2 Magnet}
    \end{minipage}
    \hfill
    \begin{minipage}[b]{0.3\textwidth}
        \includegraphics[width=\textwidth]{./ReportImages/Nabla_Magnet.png}
        \caption{Nabla Magnet\\ (Source:Valeo)}
        \label{fig:Nabla Magnet}
    \end{minipage}
\end{figure}

% \begin{figure}[h]
%     \centering
%     \includegraphics[width=0.25\textwidth]{./ReportImages/1V_Magnet.png}
%     \caption{V1 Magnet(Source:Valeo)}
%     \label{fig:V1 Magnet}
% \end{figure}

% \begin{figure}[h]
%     \centering
%     \includegraphics[width=0.25\textwidth]{./ReportImages/2V_Magnet.png}
%     \caption{V2 Magnet(Source:Valeo)}
%     \label{fig:V2 Magnet}
% \end{figure}

% \begin{figure}[h]
%     \centering
%     \includegraphics[width=0.25\textwidth]{./ReportImages/Nabla_Magnet.png}
%     \caption{Nabla Magnet(Source:Valeo)}
%     \label{fig:Nabla Magnet}
% \end{figure}

This master thesis explores a way to do surrogate modelling of the current process as is highlighted in Figure \ref{fig:EM Design Flowchart} by making use of GNN/MLP for the modelling of electrical engine designs described parameterically. \\
\begin{figure}[h]
    \centering
    \includegraphics[width=0.9\textwidth]{./ReportImages/EM_design_flowchart_v2.png} 
    \caption{EM Design Flowchart}
    \label{fig:EM Design Flowchart}
\end{figure}

The thesis is structured to follow sections namely Literature Review, Dataset, Modelling, Experiments and Results, Conclusion, Appendix and Bibliography.\\
The Literature Review section will cover the related works that has been done in this domain. In the Dataset section a detailed insight to how our data is structured is given.
In the Modelling section, the methodologies used to tackle the problem will be further elaborated on. The Experiments and Results chapter gives an outlook on the outcomes of our work in addition to other findings we unearth. 
Conclusion chapter summarizes the thesis briefly and would also give a small glimpse into areas of improvement. Finally the Bibliography section lists out the articles cited for this thesis.\\
\newpage 

\chapter*{Literature Review} 
\addcontentsline{toc}{chapter}{Literature Review}
There has been extensive research in modeling the Electric Motor with CNN based on the images of the motor cross-section. 
However our approach is progressive in the sense that once the KPIs are predicted we would like to be able to generate the inputs and reproducing images is not known to apt given the infamous known fact that AI generated images are faulty.
Instead by generating the parameters of the motor we can be rest assured of more precise results. Hence the need to focus on the inputs as they are with their parametric description.
Existing literature also covers works on modelling this work as tabular data using MLPs. Although this is fairly good forseing the impact of generating the inverse process yet MLPs cannot necessarily learn all the intricacies within motor components.
Hence the need to better represent the data typically in the form of graphs and model Graph Neural Networks to achieve the desired results. 
There has been close to no work of GNNs in this domain. However we see progress of GNNs in molecular chemistry and social networks usecases from which we draw inspiration.

\newpage 

\chapter*{Dataset} 
\addcontentsline{toc}{chapter}{Dataset}
Valeo an automotive company has supplied the dataset consisting of close to 1500 Double V Electric Motor parameters. 
There are close to 196 parameters which comprises of the geometric, physical and simulation properties of the motor.

The geometry of a whole Double V motor is as below

\begin{figure}[h]
    \centering
    \includegraphics[width=0.5\textwidth]{./ReportImages/FullMotorv2.png} 
    \caption{Complete EM Geometry(Source:Valeo)}
    \label{fig:Full Motor}
\end{figure}

Below is the geometry of 1/8 cross-section of the same motor.

\begin{figure}[h]
    \centering
    \includegraphics[width=0.75\textwidth]{./ReportImages/EMCrosssection.jpg} 
    \caption{1/8 Motor Crossection}
    \label{fig:1/8 Motor Crossection}
\end{figure}

\newpage 

For modelling the GNN, we represent the data in the form of a heterogeneous graph with different node and edge types.\\



\textbf{Node types}

\begin{enumerate}
    \item \textbf{General}
    
    \begin{itemize}
        \item General parameters:
        \[
            r = \{r_{i}\} \quad \forall i \in \{a, r, o\} 
        \]
        
        \textit{where:
        \begin{itemize}
            \item $r_{a}$: Outer Radius of the Stator
            \item $r_{r}$: Outer Radius of the Rotor
            \item $r_{o}$: Center of the EM
        \end{itemize}}
    \end{itemize}
    
    \item \textbf{Stator}
    
    \begin{itemize}
        \item Slot windings:
        \[
            sw = \{s_{i}w_{j}\} \quad \forall i \in \{1, \dots, QSim\}, \quad \forall j \in \{1, \dots, N\} 
        \]
        
        \item Slots:
        \[
            s = \{s_{i}\} \quad \forall i \in \{1, \dots, QSim\}
        \]

        \textit{where
        \begin{itemize}
            \item Qsim : Count of slots in the Stator
            \item N : Count of copper windings per slot
        \end{itemize}}
    \end{itemize}
    
    \item \textbf{Rotor}
    
    \begin{itemize}
        \item Magnet Flux Barriers:
        \[
            v = \{v_{ij}\} \quad \forall i \in \{1, \dots, T\}, \quad \forall j \in \{1, \dots, V\}
        \]
        
        \item Magnets:
        \[
            vm = \{v_{i}m_{j}\} \quad \forall i \in \{1, \dots, T\}, \quad \forall j \in \{1, \dots, V\}
        \]
        \textit{where
        \begin{itemize}
            \item T : Topology type of the EM
            \item V : Type of Magnet
        \end{itemize}
        As Valeo only manufactures Double V magnets we consider it to be 2}
    \end{itemize}    
    
\end{enumerate}

\textbf{Edge types}

\begin{enumerate}
    \item \textbf{Angle} \\
    \textbf{Relevant Paths}
    \[
    vm--vm = \{ v_{i_{1}}m_{j_{1}} - v_{i_{2}}m_{j_{2}} \}
    \forall i_1, i_2 \in \{1, \dots, T\}, \quad \forall j_1, j_2 \in \{1, \dots, V\} \mid
    i_1 = i_2, \quad j_1 \neq j_2
    \]

    \textbf{angle}=vm-vm

    \item \textbf{Distance} \\
    \textbf{Relevant Paths}
    % \[
    %     v-v = \{ (v_{i_1 j_1} - v_{i_2 j_2}), \forall i_1, i_2, j_1, j_2 \in \{1, \dots, T\} \mid i_1i_2 \neq j_1j_2 \ \land (i_1 == i_2 \lor j_1 == j_2) \}
    % \]
    \[
        vi--vi = \{v_{i j_1} - v_{i j_2}\}, \forall i \in \{1, \dots, T\}, \forall j_1, j_2 \in \{1, \dots, V\} \mid  j_1 \neq j_2
    \]
    \[
        vi--vj = \{v_{i_1 j} - v_{i_2 j}\}, \forall i_1, i_2 \in \{1, \dots, T\}, \forall j \in \{1, \dots, V\} \mid  i_1 \neq i_2
    \]
    \[
        v--vm = \{v_{i j} - v_{i}m_{j}\} \forall i  \in \{1, \dots, T\}, \quad \forall j \in \{1, \dots, V\}
    \]
    \[
        v--rr = \{v_{i j} - r_{r}\}, \forall i, j  \in \{1, \dots, T\}
    \]
    \[
        o--r = \{ (o - r_{r}), (o - r_{a})\}
    \]
    \[
        rr--s = \{r_{r} - s_{i}\}, \forall i  \in \{1, \dots, QSim\}
    \]
    \[
        s--sw = \{s_{i} - s_{i}w_{j}\}, \forall i  \in \{1, \dots, QSim\}, \forall j  \in \{1, \dots, N\}
    \]
    \[
        s--ra = \{s_{i} - r_{a}\}, \forall i  \in \{1, \dots, QSim\}
    \]
    \[
        sw--sw = \{s_{i}w_{j_1} - s_{i}w_{j_2}\}, \forall i  \in \{1, \dots, QSim\}, \forall j  \in \{1, \dots, N\} \mid (j_1 == j_2-1)
    \]

    \textbf{distance} = vi--vi + vi--vj + v--vm + v--rr + o--r + rr--s + s--sw + s--ra + sw--sw

\end{enumerate}

\textbf{Node Features}

\begin{enumerate}

    \item \textbf{v} = \{lmsov, lth1v, lth2v, r1v, r11v, r2v, r3v, r4v, rmt1v, rmt4v, rlt1v, rlt4v, hav\}

    \item \textbf{vm} = \{mbv, mhv, rmagv\}

    \item \textbf{r} = \{r\}

    \item \textbf{s} = \{b\_nng, b\_nzk, b\_s, h\_n, h\_s, r\_sn, r\_zk, r\_ng, h\_zk\}

    \item \textbf{sw} = \{bhp, hhp, rhp\}
\end{enumerate}

\textbf{Path Features}

\begin{enumerate}

    \item \textbf{vm--vm} = \{deg\_phi\}

    \item \textbf{vi--vi} = \{dsm, dsmu\}

    \item \textbf{vi--vj} = \{amtrvj-amtrvi\}

    \item \textbf{v--vm} = \{lmav, lmiv, lmov, lmuv\}

    \item \textbf{v--r} = \{amtrv, dsrv\}
    
    \item \textbf{o--r} = \{r\}

    \item \textbf{rr--s} = \{airgap\}

    \item \textbf{s--sw} = \{dhphp\}
    
    \item \textbf{sw--sw} = \{dhpng\}
    
    \item \textbf{s--ra} = \{r\_a-(r\_i + h\_n + h\_zk)\}
    
\end{enumerate}
The heterogeneous graph that was constructed earlier is as below:
\begin{figure}[h]
    \centering
    \includegraphics[width=0.9\textwidth]{./ReportImages/graph.png} 
    \caption{HetGraph}
    \label{fig:Graph}
\end{figure}

\subsection*{Scaling}
\addcontentsline{toc}{subsection}{Scaling}
To scale the input features, we have used the standard scaler from the sklearn library to scale the values to be shifted such that it is of 0 mean and unit standard deviation.
The target values however are scaled with the MinMax Scaler from sklearn to be between 0 and 1.
The scalers are applied to the train validation datasets and the same scaler is used to transform the test dataset to maintain uniformity on the predictions generated.

\subsection*{Dataset splitting}
\addcontentsline{toc}{subsection}{Dataset splitting}

We have also split the dataset to have about 50 samples for test and the remaining is used for 5 fold cross validation with 80:20 split for training and validation.
Within 5 folds, we expect to cover most grounds on training and have good monitoring on the model's performance for each fold.
The best performing model is then chosen based on the evaluation metrics we have in place and are used for inference later on.

\newpage 

\chapter*{Modelling}
\addcontentsline{toc}{chapter}{Modelling}
Since we aim to predict continuous vector values, we model this task into a regression problem

As a baseline, we first train a MLP on the tabular representation of the data and work on it further to do the same with a heterogeneous GNN.
\subsection*{Multi Linear Perceptron Model}
\addcontentsline{toc}{subsection}{Multi Linear Perceptron Model}

For the Multi Linear Perceptron (MLP) model, we use a single model with input features corresponding to all the features in the tabular topology invariant representation of the data and feed forward layers.
The model architecture is designed to predict both the 2D and 3D KPIs by have 2 separate output layers for each of the KPIs. 
Since the 2D KPI's targets are relatively learnable then that of the 3D KPI's targets we have experimented with fewer feed forward layers in the former than in the latter.
RELU layers were also added in between to serve as the activation function and produce non-linearities in the model.


\subsection*{HeterogeneousGraph Neural Network Model}
\addcontentsline{toc}{subsection}{HeterogeneousGraph Neural Network Model}

We find the heterogeneous graph to be most apt for our use case with its different node and edge types. As it preserves both the structural and semantics of our data.
Heterogeneous graph Neural Networks generally work by having separate non linear functions convolve over each edge type during message computation and over each node type when aggregating the learned information.
 

\subsection*{Loss Function}
\addcontentsline{toc}{subsection}{Loss Function}

The Mean Squared Error(MSE) loss is the loss function used for our problem with the intention that the losses are minimized. We have adopted 2 methodologies to regularize the loss one each for the 2D and 3D KPI.


% Mean Squared Error (MSE) Loss
\[
\text{MSE} = \frac{1}{N} \sum_{i=1}^{N} (y_i - \hat{y}_i)^2
\]

\[
\Delta \hat{y}_i = \hat{y}_{i+1} - \hat{y}_i
\]
\subsubsection*{Loss Regularization for 2D KPI(Torque curve)}
\addcontentsline{toc}{subsection}{Loss Regularization for 2D KPI(Torque curve)}

To smoothen out the curve for the 2D KPI(Torque curve) we apply a loss regularisation factor to take into account before backpropagating it the model during training.
We observed the curve inherently follows a decreasing pattern and hence using this knowledge penalize the loss for non-decreasing values within each prediction.


% Regularization Term

\[
\text{violations}_i = \text{ReLU}((\Delta \hat{y}_i)^2) = \text{ReLU}((\hat{y}_{i+1} - \hat{y}_i)^2)
\]
\[
\text{regularization\_term} = \frac{1}{N-1} \sum_{i=1}^{N-1} \text{violations}_i
\]

% Composite Loss
\[
\text{Y1 Loss} = \text{MSE} + \lambda_{\text{regularisation\_weight}} \times \text{regularized\_term}
\]

Theoretically, we would expect the model to generate better predictions but on closer observation we notice the curve is still not smooth.
A reason to attribute this could be the model's incapability to infer that loss decrease depends on not just the prediction and target values but also within prediction values.
Our deduction is that a single value calcuated for the entire curve may not be sufficient to regularize this loss as we imagined.

\subsubsection*{Loss Customization for 3D KPI(Efficiency Grid)}
\addcontentsline{toc}{subsection}{Loss Customization for 3D KPI(Efficiency Grid)}
The ETA Grid is a 3D plot of real numbers ranging between 0 and 100. We noticed in some portions of the grid, the plot not visible as it had nan values.
As ANN cannot be trained to predict NAN values we had these replaced to -1 and have a binary mask constructed such that values corresponding to -1 in the target have value 0 and all other values as 1.
The mask is then multiplied with both the target and the prediction. After this step, the MSE loss is calculated and backpropagated.
Mathematically, this process can be expressed as follows:
\begin{equation*}
\begin{aligned}
\text{Let } & M_{ij} = \begin{cases}
1 & \text{if } y_{ij} \neq -1 \\
0 & \text{if } y_{ij} = -1
\end{cases} \\
& \hat{y}_{ij}^{\text{masked}} = \hat{y}_{ij} \cdot M_{ij} \\
& y_{ij}^{\text{masked}} = y_{ij} \cdot M_{ij} \\
\text{{Y2 Loss}} & = MSE(\hat{y}^{\text{masked}}, y^{\text{masked}})
\end{aligned}
\end{equation*}

\textit{where
 $M_{ij}$ : mask matrix}
 \\

This formulation ensures that the -1 values (which replaced NaN values in the grid) are ignored in the loss calculation, as they are multiplied by 0 in the mask.

\[
\text{Total Loss} = \text{Y1 Loss} + \text{Y2 Loss}
\]

\subsubsection*{Evaluation Metrics}
\addcontentsline{toc}{subsection}{Evaluation Metrics}

The evaluation metrics we have considered for our regression problem is the MSE loss and the R2 score.
The model with the least loss and the highest R2 score is ideal for our application.

\newpage 

% Masking NAN values
\chapter*{Experiments and Results}
\addcontentsline{toc}{chapter}{Experiments and Results}
Firstly we train a MLP on the tabular representation of the data.
We have used MinMax Scaler for normalization of both the input and targets.
MSE loss function is used for the regression problem and Adam optimizer is used for optimization.
We have the first results on the 2D KPI-Mgrenz prediction.
These results are only considering about 400 examples. The dataset was split into training-validation(80:20) excluding test examples of about 19 examples.
\begin{figure}[h]
    \centering
    \begin{minipage}[b]{0.3\textwidth}
        \includegraphics[width=\textwidth]{./ReportImages/train_loss_y1.png}
        \caption{Training Loss for Torque Curve}
        \label{fig:Training Loss for Torque Curve}
    \end{minipage}
    \hfill
    \begin{minipage}[b]{0.3\textwidth}
        \includegraphics[width=\textwidth]{./ReportImages/train_loss_y2.png}
        \caption{Training Loss for ETA grid}
        \label{fig:Training Loss for ETA grid}
    \end{minipage}
\end{figure}

\begin{figure}[h]
    \centering
    \begin{minipage}[b]{0.3\textwidth}
        \includegraphics[width=\textwidth]{./ReportImages/train_r2_y1.png}
        \caption{Training R2 Score for Torque Curve}
        \label{fig:Training R2 Score for Torque Curve}
    \end{minipage}
    \hfill
    \begin{minipage}[b]{0.3\textwidth}
        \includegraphics[width=\textwidth]{./ReportImages/train_r2_y2.png}
        \caption{Training R2 Score for ETA grid}
        \label{fig:Training R2 Score for ETA grid}
    \end{minipage}
\end{figure}

From the training plots we see that the model has converged after having run for 10 epochs with a learning rate of 0.0075.

\begin{figure}[h]
    \centering
    \begin{minipage}[b]{0.3\textwidth}
        \includegraphics[width=\textwidth]{./ReportImages/val_loss_y1.png}
        \caption{Validation Loss for Torque Curve}
        \label{fig:Validation Loss for Torque Curve}
    \end{minipage}
    \hfill
    \begin{minipage}[b]{0.3\textwidth}
        \includegraphics[width=\textwidth]{./ReportImages/val_loss_y2.png}
        \caption{Validation Loss for ETA grid}
        \label{fig:Validation Loss for ETA grid}
    \end{minipage}
\end{figure}

\begin{figure}[h]
    \centering
    \begin{minipage}[b]{0.3\textwidth}
        \includegraphics[width=\textwidth]{./ReportImages/val_r2_y1.png}
        \caption{Validation R2 Score for Torque Curve}
        \label{fig:Validation R2 Score for Torque Curve}
    \end{minipage}
    \hfill
    \begin{minipage}[b]{0.3\textwidth}
        \includegraphics[width=\textwidth]{./ReportImages/val_r2_y2.png}
        \caption{Validation R2 Score for ETA grid}
        \label{fig:Validation R2 Score for ETA grid}
    \end{minipage}
\end{figure}

The results of the MLP model from inference is as below:

\begin{figure}[h]
    \centering
    \includegraphics[width=0.75\textwidth]{./ReportImages/kpi2d_predictions.png} 
    \caption{MLP Training Results for 2D KPI(Mgrenz)} 
    \label{fig:MLP Training Results for 2D KPI(Mgrenz)}
\end{figure}

\begin{figure}[h]
    \centering
    \includegraphics[width=0.75\textwidth]{./ReportImages/kpi3dprediction1.png} 
    \caption{1st MLP Training Results for 3D KPI(ETA)} 
    \label{fig:1st MLP Training Results for 3D KPI(ETA)}
\end{figure}

\begin{figure}[h]
    \centering
    \includegraphics[width=0.75\textwidth]{./ReportImages/kpi3dprediction2.png} 
    \caption{2nd MLP Training Results for 3D KPI(ETA)} 
    \label{fig:2nd MLP Training Results for 3D KPI(ETA)}
\end{figure}

\begin{figure}[h]
    \centering
    \includegraphics[width=0.75\textwidth]{./ReportImages/kpi3dprediction3.png} 
    \caption{3rd MLP Training Results for 3D KPI(ETA)} 
    \label{fig:3rd MLP Training Results for 3D KPI(ETA)}
\end{figure}

\begin{figure}[h]
    \centering
    \includegraphics[width=0.75\textwidth]{./ReportImages/kpi3dprediction4.png} 
    \caption{4th MLP Training Results for 3D KPI(ETA)} 
    \label{fig:4th MLP Training Results for 3D KPI(ETA)}
\end{figure}

\begin{figure}[h]
    \centering
    \includegraphics[width=0.75\textwidth]{./ReportImages/kpi3dprediction5.png} 
    \caption{5th MLP Training Results for 3D KPI(ETA)} 
    \label{fig:5th MLP Training Results for 3D KPI(ETA)}
\end{figure}


We see a good fit of the model to the data with R2 score approaching close to 1.
We have also enabled saving the trained model locally so it can be loaded on demand by the client to run inference.
\newpage 

\chapter*{Conclusion}

\newpage 

\newpage 

\listoffigures

\newpage 

\newpage 

\listoftables

\newpage 

\newpage 

\chapter*{Appendix}
\addcontentsline{toc}{chapter}{Appendix}

\newpage 

\newpage 

\chapter*{Bibliography}
\addcontentsline{toc}{chapter}{Bibliography}
\newpage 

\newpage 

\chapter*{Declaration on oath}
\addcontentsline{toc}{chapter}{Declaration on oath}

\vspace{1cm}

\noindent I hereby certify that I have written my master thesis independently and have not yet submitted it for examination purposes elsewhere. All sources and aids used are listed, literal and meaningful quotations have been marked as such.

\vspace{3cm}
\hfill\rule{15cm}{0.4pt} % Horizontal line for the signature aligned to the right

\begin{center}
    Lilly Abraham K64889, 11.12.2024 % Placeholder for the signature and date
\end{center}

\newpage 

\chapter*{Consent to Plagiarism Check}
\addcontentsline{toc}{chapter}{Consent to Plagiarism Check}
\vspace{1cm}

\noindent I hereby agree that my submitted work may be sent to PlagScan (www.plagscan.com) in digital form for the purpose of checking for plagiarism and that it may be temporarily (max. 5 years) stored in the database maintained by PlagScan as well as personal data which are part of this work may be stored there.

\vspace{0.5cm}

\noindent Consent is voluntary. Without this consent, the plagiarism check cannot be prevented by removing all personal data and protecting the copyright requirements. Consent to the storage and use of personal data may be revoked at any time by notifying the faculty.


\vspace{3cm}
\hfill\rule{15cm}{0.4pt} % Horizontal line for the signature aligned to the right

\begin{center}
    Lilly Abraham K64889, 11.12.2024 % Placeholder for the signature and date
\end{center}

\end{document}
